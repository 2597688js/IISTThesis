\chapter{Introduction}\label{ch:introduction}

Sample code for including figures, tables, algorithms, and citations
are listed here.

\section{Including Figures}\label{sec:figures}
\begin{figure}[h]
	\centering
	\includegraphics[scale=0.15]{figures/latexlogo.pdf}
	\caption{Sample figure}
	\label{fig:latexlogo}
\end{figure}


\section{Including Algorithms}\label{sec:algorithms}
\begin{algorithm}[h]
	%\SetAlgoVlined
	\caption{Sum of $N$ numbers}
	$S = 0$\\
	\For{$i=1:N$}
	{
		$S = S + i$
	}
	Ouput $S$
\end{algorithm}


\section{Including Tables}\label{sec:tables}
In this section, Table~\ref{tab:sample1} is explained.
\begin{table}[h]
	\centering
	\caption{Sample table}
	\begin{tabular}{| c | c | c |}\hline
	\textbf{Parameter} & \textbf{x} & \textbf{y} \\\hline\hline
	ABC & 2 & 4 \\\hline
	DEF & 3 & 9 \\\hline	
	\end{tabular}
	\label{tab:sample1}
\end{table}

\section{Theorem, Proof, Lemma, Corollary, Proposition, and Conjecture}
\label{sec:theorem}
\begin{theorem}
  \label{thm:th1}
  This is my first theorem.
\end{theorem}

\begin{proof}
  \label{prf:pr1}
  This is my proof.
\end{proof}

\begin{lemma}
  \label{lem:lem1}
  This is a content for sample lemma. 
\end{lemma}

\begin{corollary}
  \label{cor:cor1}
  This is a sample corollary.
\end{corollary}

\begin{proposition}
  \label{prp:prop1}
  This is an example of proposition.
\end{proposition}

\begin{conjecture}
  \label{cnj:conj1}
  This is an example of conjecture
\end{conjecture}

\section{Definition, Condition, Assumptions, Examples, and Problems}
\label{sec:defn}

\begin{definition}
  \label{def:defn1}
  An example of a definition.
\end{definition}

\begin{condition}
  \label{con:cond1}
  An example of a condition.
\end{condition}

\begin{assumption}
  \label{asm:assum1}
  You assumptions can be placed here.
\end{assumption}

\begin{example}
  \label{exm:example1}
  This is an example.
\end{example}

\begin{problem}
  \label{pbl:problem1}
  Problem statements can be put here.
\end{problem}

\section{Remarks, Claims, and Notes}
\label{sec:remarks}

\begin{remark}
  \label{rem:remark1}
  Your remarks can be written using this environment.
\end{remark}

\begin{claim}
  \label{clm:claim1}
  Claims can be made using claim environment.
\end{claim}

\begin{note}
  An example note.
\end{note}


\section{Citations}\label{sec:citations}
Sample citation~\cite{lamport1994latex}.

\section{Indexing}\label{sec:indexing} \LaTeX\ \index{LaTeX@\LaTeX} is a 
type setting system written in \TeX\ language. \LaTeX\ 
is a free software\index{Free software} originally developed by 
Leslie Lamport in 1980s. 



